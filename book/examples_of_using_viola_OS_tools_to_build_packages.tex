\chapter{Примеры использования инструментов ОС Альт для сборки пакетов}

В текущих разделах представлен материал с практическими примерами по сборке пакетов. Простыми шагами разберём подготовку системы к работе с инструментами, описанными в предыдущих главах пособия. Ниже следуют примеры подготовки и сборки \Sys{rpm}-пакета инструментами \Sys{rpmbild}, \Sys{gear} и \Sys{hasher}.

\begin{itemize}
	\item \textbf{Пример №1. Сборка пакета с помощью \Sys{rpmbuild}}.
	
	Подготовьте дерево каталогов для сборки пакета с помощью \Sys{rpmbuild}. Сформируйте \Sys{.tar}-архив с исходным кодом программы и загрузите в каталог \Sys{SOURCES}. Назовите архив в соответствии с именем \Sys{SPEC}-файла. В каталог \Sys{SPECS} загрузите \Sys{spec}-файл. 
	
	\begin{verbatim}
		$ cd ~/RPM
		$ tree
		.
		|-- BUILD
		|   `-- HelloUniverse-1.0
		|-- RPMS
		|   |-- noarch
		|   `-- x86_64
		|-- SOURCES
		|   `-- HelloUniverse-1.0.tar
		|-- SPECS
		|   `-- HelloUniverse.spec
		`-- SRPMS
	\end{verbatim}
	
	Для сборки \Sys{rpm}-пакета выполните команду: 
	\begin{verbatim}
		$ rpmbuild -bb ./SPECS/HelloUniverse.spec 
		...
		$ ls ./RPMS/x86_64/
		HelloUniverse-1.0-alt1.x86_64.rpm
		HelloUniverse-debuginfo-1.0-alt1.x86_64.rpm
	\end{verbatim}
	
	\item \textbf{Пример №2. Сборка из \Sys{.src.rpm} с помощью \Sys{hasher}.}
	
	Возьмите готовый файл \Sys{.src.rpm} и выполните команду:
	\begin{verbatim}
		$ hsh -v HelloUniverse-1.0-alt1.src.rpm  --no-sisyphus-check
	\end{verbatim} 
	
	Результат успешного завершения сборки: 
	\begin{verbatim}
		$ ls ~/hasher/repo/x86_64/RPMS.hasher/
		HelloUniverse-1.0-alt1.x86_64.rpm
		HelloUniverse-debuginfo-1.0-alt1.x86_64.rpm
	\end{verbatim}
	
	\item \textbf{Пример №3. Сборка с помощью \Sys{gear-hsh}.}
	
	Выполните команду: 
	\begin{verbatim}
		$ gear-hsh -v --no-sisyphus-check 
	\end{verbatim}
	
	Результат успешного завершения сборки --- собранный \Sys{rpm}-пакет и \Sys{.src.rpm}:
	\begin{verbatim}
		$ ls ~/hasher/repo/x86_64/RPMS.hasher/
		HelloUniverse-1.0-alt1.x86_64.rpm
		HelloUniverse-debuginfo-1.0-alt1.x86_64.rpm
		
		$ ls ~/hasher/repo/SRPMS.hasher/
		HelloUniverse-1.0-alt1.src.rpm
	\end{verbatim} 
	
	Установите собранный пакет командой: 
	\begin{verbatim}
		$ sudo apt-get install ~/hasher/repo/x86_64/RPMS.hasher/\
		HelloUniverse-1.0-alt1.x86_64.rpm
	\end{verbatim}
\end{itemize} 


\section{Подготовка пространства}
Соберем в \Sys{rpm}-пакет собственную программу. В каталоге сохраним файлы с исходным кодом --- \Sys{HelloUniverse.cpp} и \Sys{Makefile}.
\begin{verbatim}
	$ ls -la
	HelloUniverse.cpp
	Makefile
\end{verbatim}

\Emph{Подготовка окружения}

Инструменты, необходимые для сборки: 
\begin{itemize}
	\item система контроля версий Git;
	\item текстовый редактор (Vim, Emacs, MC);
	\item \Sys{rpmbuild};
	\item \Sys{Hasher};
	\item \Sys{GEAR}.
\end{itemize}

Для работы с \Sys{rpmbuild} установите также сборочные зависимости:
\begin{itemize}
	\item make;
	\item gcc-c++.
\end{itemize} 

После установки программ: 
\begin{enumerate}
	\item Настройте Git;
	\item Создайте \Sys{gpg}-ключ подписи для работы с \Sys{gear};
	\item Настройте окружение \Sys{RPM};
	\item Настройте окружение \Sys{Hasher}.
\end{enumerate}


\section{Написание \Sys{spec}-файла и правил \Sys{GEAR}}
Подготовка \Sys{gear}-репозитория.

Перейдите в каталог с исходным кодом и проинициализируйте репозиторий: 
\begin{verbatim}
	$ cd HelloUniverse/
	$ git init	
\end{verbatim}

Создайте каталог \Sys{.gear}, перейдите в него и создайте файлы \Sys{HelloUniverse.spec} и \Sys{rules}:
\begin{verbatim}
	$ mkdir .gear
	$ cd .gear
	$ touch HelloUniverse.spec
	$ touch rules
\end{verbatim} 

Заполните \Sys{spec}-файл \Sys{HelloUniverse.spec}:
\begin{verbatim}
	%define _unpackaged_files_terminate_build 1
	
	Name:       HelloUniverse
	Version:    1.0
	Release:    alt1
	Summary:    Most simple RPM package
	License:    no
	Group:	Development/Other
	Source: %name-%version.tar
	BuildRequires: gcc-c++
	%description
	This is my first RPM
	
	%prep
	%setup
	
	%build
	%make_build HelloUniverse
	
	%install
	mkdir -p %{buildroot}%_sbindir
	install -m 755 HelloUniverse %{buildroot}%_sbindir
	
	%files
	%_sbindir/%name
	
	%changelog
\end{verbatim} 

В файл \Sys{rules} пропишите:
\begin{verbatim}
	tar: . name=@name@-@version@ base=@name@-@version@
	spec: .gear/HelloUniverse.spec
\end{verbatim} 

Перейдите на уровень выше, в корневой каталог проекта, и сделайте первый коммит: 
\begin{verbatim}
	$ cd ../ 
	$ git add HelloUniverse.cpp
	$ git add Makefile
	$ git add .gear/HelloUniverse.spec
	$ git add .gear/rules
	$ git commit -m "first commit"
\end{verbatim}

Теперь у нас есть готовая к сборке в \Sys{rpm}-пакет программа. 

После того, как мы убедимся, что программа собирается в пакет и работает, рекомендуется написать \Sys{changelog}, сделать новый коммит с релизом и поставить тег. 


\section{Описание пакета с исходными текстами на \Sys{C++}}
Содержимое файла \Sys{HelloUniverse.cpp}.
\begin{verbatim}
	#include <iostream>
	int main()
	{
		std::cout << "Hello Universe\n";
		return 0;
	}
\end{verbatim} 

Содержимое файла \Sys{Makefile}.
\begin{verbatim}
	HelloUniverse: HelloUniverse.cpp
	g++ ./HelloUniverse.cpp -o HelloUniverse
	clear:
	rm ./HelloUniverse
\end{verbatim}} 