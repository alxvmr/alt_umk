\hypertarget{4}{\chapter{Инструмент GEAR}}\label{chapter-gear}
\Emph{GEAR (Get Every Archive from git package Repository)} ---
инструмент для подготовки, поддержки и сборки пакетов в \Sys{git}-репозитории%
\footnote{\href{https://www.altlinux.org/Gear}{https://www.altlinux.org/Gear}}%
\footnote{\href{https://www.altlinux.org/Gear/\%D0\%A1\%D0\%BF\%D1\%80\%D0\%B0\%D0\%B2\%D0\%BE\%D1\%87\%D0\%BD\%D0\%B8\%D0\%BA}{https://www.altlinux.org/Gear/Справочник}}.

\marginalia{ex_sign_col}{Для работы с текущим разделом необходимо изучить
документацию по работе с распределённой системой управления версиями \Sys{Git}.}

\Emph{Git} --- система контроля версий, предназначенная для хранения файлов, истории
их изменений и служебной информации.

\Emph{\Sys{git}-репозиторий} --- набор файлов, находящийся под управлением системы
контроля версий \Sys{Git}.

\Emph{\Sys{gear}-репозиторий} ---  \Sys{git}-репозиторий, дополненный \Sys{gear}
 инструкциями для подготовки исходных данных для последующей сборки пакета.

%Добавить достоинства gear конспект
\begin{itemize}
	\item Расширенный набор утилит упрощает процессы сборки пакетов на всех этапах работы с исходными данными.
	\item \Sys{GEAR} интегрирован с инструментами \Sys{rpmbuild} и \Sys{hasher}.
	\item Работа в  \Sys{git}-репозитории добавляет плюсы системы контроля версий и предоставляет возможность
	воспроизводимой сборки пакета.
	\item  \Sys{GEAR} даёт возможности гибкой настройки репозитория исходных данных.
\end{itemize}

Для начала работы необходимо установить пакет \Sys{gear}:
\begin{verbatim}
# apt-get install gear
\end{verbatim}

Подробная инструкция базовых компонентов:
 \begin{verbatim}
$ man gear
\end{verbatim}

\begin{verbatim}
$ man gear-rules
\end{verbatim}

\section{Описание GEAR}
При сборке пакетов \Sys{gear} предлагает работать в том же \Sys{git}-репозитории,
в котором хранятся исходные данные пакета. \Sys{GEAR} позволяет целиком импортировать
историю разработки, предоставляет различные средства импорта исходных данных и
различные варианты организации репозитория.

Идея \Sys{GEAR} в доступности всего необходимого для сборки пакета в \Sys{git}-репозитории
либо в репозитории пакетов, на основе которого ведётся сборка. Система контроля версий
\Sys{Git} предоставляет встроенные механизмы обеспечения целостности (контрольные суммы,
криптографически подписанные теги) для задач управления пакетами. Появляется возможность
воспроизводимой собрать <<такой же>> пакет ещё раз, опираясь на логически
законченную версию изменений, зафиксированную на момент времени%
\footnote{коммит (commit) --- законченная версия изменений в терминологии git.}.

Для удобства сборки пакетов, \Sys{gear} интегрирован с инструментами \Sys{rpm-build} и
\Sys{hasher}. Из \Sys{gear}-репозитория можно одной командой собрать пакет при помощи
\Sys{rpmbuild} или \Sys{hsh}.

Сборка пакета ведётся из конкретного коммита, который мы в дальнейшем будем называть главным.
Именно в нём должна находится нужная версия \Sys{.spec}-файла и правил экспорта. \Sys{GEAR}
позволяет экспортировать из \Sys{git}-репозитория каталоги и подкаталоги в виде архивов,
экспортировать отдельные файлы, вычислять разницу (\Sys{diff}) и сохранять её в виде патча.
При этом может использоваться как состояние в главном коммите, так и в любом из его предков,
прямых и не прямых. Это позволяет гибко и удобно организовать работу по поддержке пакета.
Все нужные исходные данные можно хранить в \Sys{git} так, чтобы с ними было удобно работать,
а затем экспортировать так, чтобы ими было удобно воспользоваться из \Sys{.spec}-файла.

\section{Правила экспорта}

Правила экспорта для \Sys{GEAR} описываются в текстовом файле, который также хранится в репозитории.
По умолчанию \Sys{GEAR} ищет этот файл по пути \Sys{.gear/rules} или \Sys{.gear-rules}, от корня репозитория,
в главном коммите.

Этот файл состоит из одной или нескольких строк, каждая из которых имеет следующий формат: \verb!<директива>: <параметры>!.

Параметры разделяются пробельными символами.
Пустые строки и строки, начинающиеся с <<\#>>, игнорируются.

В значениях многих параметров и опций директив могут применяться ключевые слова:
\begin{itemize}
	\item \Sys{@name@} --- будет заменено на имя пакета (извлекается из \Sys{.spec}-файла);
	\item \Sys{@version@} --- будет заменено на версию пакета (извлекается из \Sys{.spec}-файла);
	\item \Sys{@release@} --- будет заменено на релиз пакета (извлекается из \Sys{.spec}-файла).
\end{itemize}

\Emph{Теги и пути}

По умолчанию все пути в аргументах директив считаются от корня репозитория главного коммита.
Однако большинство директив позволяет указать другой коммит в качестве основы. Для этого путь
должен быть передан в формате:
\begin{verbatim}
base_tree:path_to_file.
\end{verbatim}

В качестве \Sys{base\_tree} может выступать:
\begin{itemize}
	\item полный идентификатор коммита (\Sys{SHA-1}, 40 шестнадцатиричных цифр);
	\item имя тега \Sys{GEAR};
	\item символ <<.>>, обозначающий главный коммит.
\end{itemize}

В любом случае коммит, на который так ссылаются, должен быть предком главного коммита.

\Emph{Основные директивы}

Ниже приведены основные директивы \Sys{GEAR} и их аргументы. Подробнее с ними можно
ознакомиться в \Sys{man gear-rules}.

\begin{itemize}
	\item \Emph{\Sys{spec: <путь> }}
	
	Задаёт путь к \Sys{.spec}-файлу. По умолчанию \Sys{GEAR} использует файл с
		расширением \Sys{.spec} из корня репозитория в главном коммите,
		если такой файл там только один. Единственный аргумент --- путь к \Sys{.spec}-файлу.
	
	\item \Emph{\Sys{copy: <glob>\ldots}}
	Скопировать файл, соответствующий указанному шаблону поиска (glob pattern).
		Может принимать несколько аргументов, для каждого из которых должны
		быть найдены соответствующие файлы.
	
	Также существуют директивы \Sys{gzip}, \Sys{bzip2}, \Sys{lzma}, \Sys{lzma}, \Sys{zstd},
		аналогичные \Sys{copy}, но сжимающие экспортированный файл подходящим алгоритмом сжатия.
	\item  \Emph{\Sys{tar: <tree\_path>}}

	Экспортировать каталог из репозитория в виде \Sys{tar}-архива. Допустимые опции:
	\begin{itemize}
		\item \Sys{name=<archive\_name>} имя архива (без суффикса \Sys{.tar});
		\item \Sys{base=<base\_name>} внутри архива будет создан каталог с указанным именем
			и все файлы будут помещены в него;
		\item \Sys{suffix=<suffix>} расширение создаваемого архива (по умолчанию --- \Sys{.tar});
		\item \Sys{exclude=<glob\_patter>} не включать в архив файлы, соответствующие указанному шаблону поиска.
	\end{itemize}
	
	Помимо стандартных ключевых слов, в опциях \Sys{name} и \Sys{base} может применяться ключевое
		слово \Sys{@dir@}, которое будет заменено на имя каталога из параметра \Sys{tree\_path}.
	\item \Emph{\Sys{zip: <tree\_path>}}
	
	Экспортировать каталог из репозитория в виде \Sys{zip}-архива. Принимает те же аргументы,
		что и директива \Sys{tar}, использует \Sys{.zip} в качестве расширения по умолчанию.
	
		Также существуют директивы \Sys{tar.gz}, \Sys{tar.bz2}, \Sys{tar.lzma}, \Sys{tar.xz}, \Sys{tar.zst}, аналогичные
		\Sys{tar}, но сжимающие созданный архив подходящим алгоритмом сжатия. Чаще всего используются
		несжатые архивы, так как сжатия, используемого при сборке \Sys{SRPM}, обычно достаточно.
	\item  \Emph{\Sys{diff: <old\_tree\_path> <new\_tree\_path>}}
	
	Создать \Sys{unified diff} между указанными каталогами и сохранить его в виде патча. Допустимые опции:
	\begin{itemize}
		\item \Sys{name=<diff\_name>} имя создаваемого файла;
		\item \Sys{exclude=<glob\_patter>} игнорировать файлы, соответствующие указанному шаблону поиска.
	\end{itemize}
	
	Помимо стандартных ключевых слов в опции \Sys{name} могут применяться ключевые слова \Sys{@old\_dir@}
		и \Sys{@new\_dir@}, которые будут заменены на имя каталога из параметра \Sys{old\_tree\_path}
		и \Sys{new\_tree\_path} соответственно.
\end{itemize}

Все приведённые директивы требуют, чтобы все указанные в них файлы и каталоги существовали. Если какого-то
файла или каталога не будет существовать, экспорт завершится ошибкой. Однако существуют аналогичные им директивы,
заканчивающиеся знаком вопроса (например, \Sys{tar?:} или \Sys{copy?:}), которые игнорируют отсутствующие файлы.
Это может быть удобно, чтобы не приходилось менять правила экспорта при добавлении и удалении патчей.

\section{Основные типы устройства \Sys{gear}-репозитория}
Гибкость \Sys{GEAR}\footnote{\href{https://www.altlinux.org/\%D0\%A0\%D1\%83\%D0\%BA\%D0\%BE\%D0\%B2\%D0\%BE\%D0\%B4\%D1\%81\%D1\%82\%D0\%B2\%D0\%BE_\%D0\%BF\%D0\%BE_gear}{https://www.altlinux.org/Руководство\_по\_gear}}
означает, что каждый пользователь может настроить его по своему усмотрению. Существует несколько
распространённых способов организации \Sys{gear}-репозитория в качестве основы для более сложных конфигураций.
Знакомство с ними поможет понять организацию репозитория при совместной работе над пакетами.

Базовые виды ведения \Sys{GEAR} репозиториев:
\begin{itemize}
	\item \Emph{Архив с исходными данными и патчи.}
	
	Исходные данные хранятся в каталоге \Sys{package\_name}; дополнительные изменения хранятся в виде патчей.
		Подобные репозитории создаёт команда \Sys{gear-srpmimport}, и также выглядят импортированные
		пакеты в \Sys{git.altlinux.org/srpms}.
	
	В каталоге \Sys{package\_name} принято хранить не изменённые исходные данные, для их обновления удобно
		использовать команду \Sys{gear-update}.
	
	Такой формат будет больше всего знаком пользователям без опыта поддержки пакетов \Sys{source RPM}
		и при работе с проектами, не имеющими публичного \Sys{git}-репозитория или не использующими \Sys{git}.
	
	Пример \Sys{.gear/rules}:
\begin{verbatim}
        tar: package_name
        copy?: *.patch
\end{verbatim}
	
	\item \Emph{Репозиторий с историей исходного репозитория и модифицированными исходными данными.}
	
	Создаётся копия исходного \Sys{git}-репозитория. Находится коммит, из которого нужно взять исходные
	данные для сборки. На основе этого коммита добавляется каталог \Sys{.gear} и \Sys{.spec}-файл.
	При обновлении новые исходные данные сливаются (\Sys{merge}) в эту ветку. Создаётся \Sys{gear}-тег,
	соответствующий собираемой версии. Изменения могут храниться в виде патчей, но чаще вносятся в
	текущую ветку со \Sys{spec}-файлом, затем изменения экспортируются в виде одного большого патча.
	
	Пример \Sys{.gear/rules} с генерацией патча:
\begin{verbatim}
        tar: v@version@:.
        diff: v@version@:. . exclude=.gear/** exclude=*.sp
\end{verbatim}
	
	\item \Emph{Пустая ветка со \Sys{.spec}-файлом и отдельная история разработки.}
	
	Изначально \Sys{.spec}-файл и \Sys{.gear} ведутся в отдельной ветке, содержащей главный коммит.
		Исходные данные, необходимые для сборки пакета, сливаются (\Sys{merge}) при необходимости
		с \Sys{git} стратегией \Sys{ours} таким образом, нужные коммиты оказываются в истории
		главного, но сами исходные данные в рабочее дерево каталога не попадают. Если нужно внести
		какие-то специфичные изменения, они вносятся в отдельную ветку, например \Sys{alt-fixes}.
		При каждом обновлении кода ветка \Sys{alt-fixes} и соответствующий ей \Sys{GEAR}-тег должны
		обновляться. В нашем примере \Sys{GEAR}-тег совпадает с именем ветки.
	
	Пример \Sys{.gear/rules}:
\begin{verbatim}
        tar: v@version@:.
        diff: v@version@:. alt-fixes:.
\end{verbatim}
\end{itemize}

\section{Работа с GEAR}

\subsection*{Импорт \Sys{.src.rpm}}

Пакет в формате \Sys{source RPM} можно импортировать в \Sys{gear}-репозиторий командой \Sys{gear-srpmimport}:
\begin{verbatim}
$ mkdir package_name
$ cd package_name
$ git init -b sisyphus
$ gear-srpmimport /путь/к/package_name.src.rpm
\end{verbatim}

\subsection*{Получение готового репозитория с внешнего git-сервера}

Достаточно клонировать репозиторий:
\begin{verbatim}
$ git clone <repository url> package_name
$ cd package_name
\end{verbatim}
Никаких дополнительных настроек не требуется.

\subsection*{Сборка пакета}

Основным инструментом экспорта и сборки пакетов является команда \Sys{gear}. На практике удобно
использовать предоставляемые команды-обёртки, а к самой команде \Sys{gear} прибегать только в
самых сложных случаях. Командной обёртке можно передавать опции как утилиты \Sys{gear}, так и
вложенной утилиты.

Командная обёртка \Sys{gear} для \Sys{rpmbuild} --- \Sys{gear-rpm}. Для сборки пакета используйте команду:
\begin{verbatim}
$ gear-rpm ----verbose -ba
\end{verbatim}


Собрать пакет при помощи \Sys{hsh}:
\begin{verbatim}
$ gear-hsh ----verbose
\end{verbatim}

Команде \Sys{gear-hsh} можно передать как аргументы \Sys{gear}, так и аргументы \Sys{hsh}.

Если не указана опция \Sys{----tree-ish}, \Sys{gear}, в качестве главного коммита использует текущий
коммит (HEAD). Если хочется проверить свежие изменения без создания коммита, можно использовать опцию
\Sys{----commit}. Опция \Sys{----commit} доступна и для \Sys{gear}, и для \Sys{gear-rpm}, и для \Sys{gear-hsh}. В таком случае
будет создан временный коммит (аналогично \Sys{git commit -a}), и пакет будет собран уже из него.
Стоит отметить, что, аналогично \Sys{git commit -a}, в таком коммите не будет новых файлов, если они
ещё не были добавлены в \Sys{git} командой \Sys{git add}.

\subsection*{Сборка пакета из репозитория разработчиков}

Для примера возьмём конкретный пакет, который уже есть в репозитории Сизиф, например \Sys{pixz}, и попробуем
собрать его с нуля.

Для начала клонируем репозиторий. Сразу зададим имя удалённого репозитория:
\begin{verbatim}
$ git clone https://github.com/vasi/pixz -o upstream
$ cd pixz
\end{verbatim}

Перейдём в ветку, из которой будем собирать пакет:
\begin{verbatim}
$ git checkout -b sisyphus upstream/master
\end{verbatim}

Переместимся на тег (тег --- это ссылка, указывающая на определённый \Sys{git}-коммит в репозитории),
соответствующий версии, которую мы будем собирать. На момент написания пособия последний тег \Sys{v1.0.7}:
\begin{verbatim}
$ git reset ----hard v1.0.7
\end{verbatim}

Определим правила экспорта:
\begin{verbatim}
$ mkdir .gear
$ vim .gear/rules
\end{verbatim}

Правила зададим следующие:
\begin{itemize}
	\item \Sys{.spec}-файл перенесём в каталог \Sys{.gear}, чтобы он не путался с основными исходными данными;
	\item исходные данные будем забирать из тега \Sys{GEAR}, соответствующего апстримному;
	\item сразу создадим патч, включающий все наши изменения(каталог \Sys{.gear} в этот патч мы включать не будем).
\end{itemize}

Получаем следующий файл \Sys{.gear/rules}:
\begin{verbatim}
    spec: .gear/pixz.spec
    tar: v@version@:.
    diff: v@version@:. . exclude=.gear/**
\end{verbatim}

Создадим соответствующий версии тег \Sys{GEAR}:
\begin{verbatim}
$ gear-store-tags v1.0.7
\end{verbatim}

Напишем \Sys{.spec}-файл:
\begin{verbatim}
$ vim .gear/pixz.spec

%define _unpackaged_files_terminate_build 1

Name:     pixz
Version:  1.0.7
Release:  alt1

Summary:  Parallel, indexed xz compressor
License:  BSD-2-Clause
Group:    Other
Url:      https://github.com/vasi/pixz

Source:   %name-%version.tar
Patch:    %name-%version-%release.patch

BuildRequires: pkgconfig(liblzma) pkgconfig(libarchive)
BuildRequires: /usr/bin/a2x

%description
Pixz (pronounced *pixie*) is a parallel, indexing version of 'xz'.

The existing XZ Utils provide great compression in the '.xz' file
format, but they produce just one big block of compressed data.
Pixz instead produces a collection of smaller blocks which makes
random access to the original data possible. This is especially
useful for large tarballs.

%prep
%setup
%patch -p1

%build
%autoreconf
%configure
%make_build

%install
%makeinstall_std

%check
%make_build check

%files
%_bindir/*
%_man1dir/*
%doc *.md

%changelog
* Wed Mar 31 2021 Author Name <email@altlinux.org> 1.0.7-alt1
- Initial build for Sisyphus

\end{verbatim}

Важно, что версия в \Sys{spec}-файле --- \Sys{1.0.7}, поэтому конструкция \Sys{v@version@},
которую мы использовали в \Sys{.gear/rules}, раскроется в строку \Sys{v1.0.7} ---
именно такой тег \Sys{GEAR} мы создали.

Добавим каталог \Sys{.gear} в \Sys{git}:
\begin{verbatim}
$ git add .gear
\end{verbatim}

Можно попробовать собрать пакет:
\begin{verbatim}
$ gear-rpm ----verbose ----commit -ba
\end{verbatim}

При необходимости, можно поправить \Sys{spec}-файл и добавить нужные зависимости.
Когда пакет собирается и работа с ним закончена, стоит зафиксировать все изменения:
\begin{verbatim}
$ gear-commit -a
\end{verbatim}
Команда оформит изменения и предложит их в утверждённом сообществом формате.

Обобщим опыт, упрощающий работу с \Sys{git}-репозиториями, предназначенными для хранения исходного кода пакетов.

\begin{itemize}
	\item \Emph{Одна кодовая база --- один репозиторий.}

	Не имеет смысла хранить в одном репозитории исходные данные не связанных между собой пакетов.
		В этом правиле бывают исключения в случае, если культура разработки предполагает хранение
		разных составных частей проекта в одном репозитории.
	\item  \Emph{Храните в \Sys{git}-репозитории распакованный исходный код.}

	Исходные данные, поставляемые в виде архивов (\Sys{tar}, \Sys{zip}) и сжатые различными
		алгоритмами (\Sys{gz}, \Sys{bzip2}, \Sys{xz}), удобнее распаковать. Это упрощает
		использование средств \Sys{git} для работы с этими данными: так легче отслеживать изменения,
		ниже трафик при обновлениях и~т.\,д.
	\item \Emph{Отделяйте свои изменения от изменений исходной кодовой базы.}

	Если вы не разработчик пакета, который собираете, лучше хранить свои изменения отдельно
		от кода разработчиков, например в отдельной ветке (и формировать \Sys{diff} средствами \Sys{GEAR})
		или в виде патчей. Это заметно упрощает совместную работу над пакетом, обновления и аудит.
\end{itemize}
\section{Вопросы для самопроверки}

\begin{enumerate}
	\item Что такое инструмент \Sys{git}?
	\item Что такое инструмент \Sys{GEAR}?
	\item Где хранятся правила экспорта для \Sys{GEAR}?
	\item Перечислите правила, упрощающие работу с \Sys{git}-репозиториями?
	\item Какой формат у файла с правилами экспорта для \Sys{GEAR}?
	\item Что означает параметр \Sys{@name@} в файле \Sys{.gear-rules}?
	\item Как формируется путь в аргументах директив в файле \Sys{.gear-rules}?
	\item Перечислите основные директивы в файле \Sys{.gear-rules}?
	\item Какой командой собирается пакет с помощью \Sys{GEAR}?
	\item Как создать тег, соответствующий версии пакета с помощью \Sys{GEAR}?
\end{enumerate}
