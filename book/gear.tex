\chapter{Инструмент Gear}\label{chapter-gear}
\Emph{GEAR (Get Every Archive from git package Repository)} --- инструмент для поддержки и сборки пакетов из \Sys{git}-репозиториев\footnote{\href{https://www.altlinux.org/Gear}{https://www.altlinux.org/Gear}}\footnote{\href{https://www.altlinux.org/Gear/\%D0\%A1\%D0\%BF\%D1\%80\%D0\%B0\%D0\%B2\%D0\%BE\%D1\%87\%D0\%BD\%D0\%B8\%D0\%BA}{https://www.altlinux.org/Gear/Справочник}}.

\marginalia{ex_sign_col}{Для работы с текущим разделом необходимо изучить документацию по работе с распределённой системой управления версиями \Sys{Git}.}

\Emph{Git} --- это технология контроля версий, которая следит за изменениями в файлах. \Sys{Git}-репозиторий хранит исходный код и данные для сборки пакетов, включая история изменений, сведения о версиях, авторах изменений и прочую информацию. 

При сборке пакетов \Sys{gear} предлагает работать в том же \Sys{git}-репозитории, в котором хранятся исходные тексты пакета. \Sys{Gear} позволяет целиком импортировать историю разработки, предоставляет различные средства импорта исходных текстов и различные варианты организации репозитория. 

Идея \Sys{gear} в доступности всего необходимого для сборки пакета в \Sys{git}-репозитории, либо в репозитории пакетов, на основе которого ведётся сборка. Система контроля версий \Sys{Git} предоставляет встроенные механизмы обеспечения целостности (контрольные суммы, криптографически подписанные теги) для задач управления пакетами. Появляется возможность воспроизводимой сборки --- собрать <<такой же>> пакет ещё раз, опираясь на логически законченную версию изменений зафиксированную на момент времени --- коммит (commit) в терминологии git.

Сборка пакета ведётся из конкретного коммита, который мы в дальнейшем будем называть главным. Именно в нём должна находится нужная версия \Sys{SPEC}-файла и правил экспорта. \Sys{GEAR} позволяет экспортировать из \Sys{git}-репозитория каталоги и подкаталоги в виде архивов; экспортировать отдельные файлы; вычислять разницу (diff) и сохранять её в виде патча. При этом может использоваться как состояние в главном коммите, так и в любом из его предков, прямых и не прямых. Это позволяет гибко и удобно организовать работу по поддержке пакета: все нужные исходные тексты можно хранить в \Sys{git} так, чтобы с ними было удобно работать, а затем экспортировать так, чтобы ими было удобно воспользоваться из \Sys{SPEC}-файла.

Правила экспорта для \Sys{gear} описываются в текстовом файле, который также хранится в репозитории. По умолчанию \Sys{gear} ищет этот файл по пути \Sys{.gear/rules} или \Sys{.gear-rules}. Подробнее про синтаксис этого файла можно прочитать ниже, в разделе <<Правила экспорта>>.

Для удобства сборки пакетов \Sys{gear} интегрирован с инструментами \Sys{rpm-build} и \Sys{hasher}: из \Sys{gear}-репозитория можно одной командой собрать пакет при помощи \Sys{rpmbuild} или \Sys{hsh}. Подробнее об этом можно прочитать далее в разделе <<Быстрый старт>>. 

