\chapter*{Заключение}\addcontentsline{toc}{chapter}{Заключение}
Материал пособия должен помочь освоить настройку инфраструктуры для разработки и сборки программных пакетов. Разделы глав сформированы в соответствии с рекомендуемым порядком освоения инструментов --- от простого к сложному. 

Вначале описываются основные термины и понятия --- системы управления пакетами низкого и высокого уровня. Дается структура \Sys{rpm}-пакета и объясняются базовые методы работы с пакетами в системах Альт. Инфраструктура разработки программных пакетов и сборки программного обеспечения описана в \hyperlink{3}{3}, \hyperlink{4}{4}, \hyperlink{5}{5}, \hyperlink{6}{6} главах и включает:
\begin{itemize}
	\item пакетный менеджер \Sys{RPM};
	\item систему управления пакетами \Sys{APT};
	\item контейнер для сборки пакетов \Sys{Hasher};
	\item набор инструментов \Sys{GEAR}.
\end{itemize}

В заключительном разделе приводится практический пример подготовки программы на языке \Sys{С++} и упаковки ее в \Sys{rpm}-пакет различными способами. 