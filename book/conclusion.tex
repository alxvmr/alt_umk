\chapter*{Заключение}\addcontentsline{toc}{chapter}{Заключение}

Уважаемый читатель! 

Вы познакомились с технологиями и инструментами продукта «Альт Платформа», 
развиваемого и поддерживаемого ООО «Базальт СПО». 

<<\href{https://www.basealt.ru/alt-platform}{Альт платформа}>> --- 
технологический комплекс для сборки общесистемного 
и прикладного программного обеспечения и выпуска дистрибутивов. 

В данном пособии вы научились собирать пакеты для разных процессорных архитектур,
убедились в том, что сборка пакета для Эльбрус по технологии не отличается от сборки для 
архитектуры x86\_64. 

Если вас увлёк процесс сборки и вы хотите непосредственно участвовать в работе над свободным 
проектом <<Сизиф>> в команде ALT Linux Team или поступить 
на работу в ООО <<Базальт СПО>> в качестве сопровождающего (мантейнера) пакетов свободного ПО, 
то самый прямой путь тут начать процесс вступления в команду. Процесс описан в статье 
\href{https://www.altlinux.org/Join}{Join} на wiki сообщества. 

В качестве направления дальнейшего развития, попробуйте 
собрать свой дистрибутив. Для этого вам понадобится ещё один 
инструмент: \href{https://www.altlinux.org/Mkimage-profiles}{\Sys{mkimage-profiles}}.
Mkimage-profiles --- система управления конфигурацией семейств дистрибутивов <<ALT>> для различных платформ.
\href{https://www.basealt.ru/fileadmin/docs/License_Alt-Platform_10.pdf}{Лицензия на <<Альт Платформа>>} 
позволяет вам создавать свой дистрибутив под свободной лицензией GPL v3.

Пока готовилось пособие произошло знаменательное событие, 04.07.2024~г. \href{https://arppsoft.ru/news/arpp/16516/}{% 
	«МЦСТ» открыл исходный код} программных компонентов для архитектуры «Эльбрус».

Составители пособия выражают надежду, что материал изложен подробно и понятно, 
однако с удовольствием рассмотрят любые пожелания и замечания через форму отзывов 
на \href{https://www.basealt.ru/contacts}{сайте <<Базальт СПО>>}.
