\chapter*{Введение}\addcontentsline{toc}{chapter}{Введение}
\markboth{Введение}{Введение}

Учебно-методическое пособие <<Инфраструктура разработки программных пакетов и сборки 
программного обеспечения>> предназначено для очного и дистанционного обучения студентов 
в рамках гуманитарно-технологической платформы по программам магистратуры и 
дополнительного профессионального образования <<Информационная культура цифровой 
трансформации>> базовой кафедры общественно-государственного объединения <<Ассоциация 
документальной электросвязи>> (АДЭ) <<Технологии электронного обмена данными>> (ТЭОД) 
в Московском техническом университете связи и информатики (МТУСИ). 

Пособие состоит из введения, заключения, семи глав и содержит практикумы по следующим направлениям:
\begin{itemize}
	\item Пакетный менеджер.
	\item Основные команды пакетного менеджера \Sys{RPM}.
	\item Общая информация о сборке \Sys{RPM}-пакета.
	\item Инструмент \Sys{rpmbuild}.
	\item Инструмент \Sys{HASHER}.
	\item Инструмент GEAR.
	\item Примеры и упражнения.
\end{itemize}

Пособие включает сведения о программной платформе лабораторного практикума на базе 
отечественных операционных систем семейства <<Альт>>, представленного компанией <<Базальт СПО>>, 
единственного российского разработчика системного программного обеспечения, создавшего 
собственную технологическую среду распределённой коллективной разработки и обеспечения 
жизненного цикла программного обеспечения <<\href{https://www.basealt.ru/alt-platform}{Альт платформа}>> 
(включена в Реестр Российского Программного обеспечения\footnote{\href{https://reestr.digital.gov.ru}{Реестр Российского Программного обеспечения}}).

Разработки дистрибутивов операционных систем компании <<Базальт СПО>> основаны на 
отечественной инфраструктуре разработки <<Сизиф>> (Sisyphus), которая находится на 
территории РФ, принадлежит и поддерживается компанией <<Базальт СПО>>. В основе 
Sisyphus лежат технологии сборки компонентов системы и учёта зависимостей между ними, а
также отработанные процессы по взаимодействию разработчиков. На базе репозитория
периодически формируется стабильный репозиторий пакетов (программная платформа), которая поддерживается
в течение длительного времени и используется в качестве базы для построения дистрибутивов 
линейки <<Альт>> и обеспечения их жизненного цикла.

ООО <<Базальт СПО>> выпускает линейку дистрибутивов разного назначения для различных аппаратных архитектур.
В репозитории Sisyphus поддерживаются архитектуры: \Sys{i586}, \Sys{x86\_64}, \Sys{armh} (\Sys{armv7}), \Sys{aarch64} (\Sys{armv8}), 
Эльбрус (с третьего по шестое поколение), \Sys{riscv64}, \Sys{mipsel}, \Sys{loongarch}.

Часть дистрибутивов включена в \href{https://reestr.digital.gov.ru}{Единый реестр российских программ} для электронных 
вычислительных машин и баз данных --- это <<Альт СП>>, имеющий сертификаты ФСТЭК России, 
Минобороны России и ФСБ России, <<Альт Виртуализация>>, <<Альт Сервер>>, <<Альт Рабочая станция>>, 
<<Альт Образование>>, так и другие, бесплатные и свободные: Simply Linux, различные стартеркиты 
(\Sys{starterkits}) и регулярные (\Sys{regular}) сборки. Дистрибутив --- это составное произведение, 
в составе которого есть программа для дистрибуции (установки), называемая инсталлятор, и 
набор системного и прикладного ПО. В основе всех дистрибутивов лежат пакеты свободного 
программного обеспечения.

Свободное программное обеспечение (СПО) --- это программное обеспечение, распространяемое 
на условиях простой (неисключительной) лицензии, которая позволяет пользователю:
\begin{enumerate}
	\item использовать программу для ЭВМ в любых, не запрещённых законом, целях;
	\item получать доступ к исходным текстам (кодам) программы как в целях изучения и адаптации, 
	так и в целях переработки программы для ЭВМ; распространять программу (бесплатно или за плату, по своему усмотрению);
	\item вносить изменения в программу для ЭВМ (перерабатывать) и распространять экземпляры изменённой (переработанной) 
	программы с учётом возможных требований наследования лицензии;
	\item в отдельных случаях (copyleft лицензия) распространять модифицированную компьютерную программу пользователем 
	на условиях, идентичных тем, на которых ему предоставлена исходная программа.
\end{enumerate}

Примерами свободных лицензий являются:
\begin{enumerate}
	\item \href{https://www.gnu.org/licenses/gpl-3.0.html}{\Sys{GNU general public license}}. Version 3, 29 June 2007 (Стандартная общественная лицензия GNU. Версия 3, от 29 июня 2007 г.).
	\item \href{https://en.wikipedia.org/wiki/BSD_licenses}{\Sys{BSD license}}, New Berkley Software Distribution license (Модифицированная программная лицензия университета Беркли).
\end{enumerate}

СПО отлично подходит для целей обучения и для разработки собственных решений, потому что весь 
код доступен для изучения и модификации. Однако авторы настоятельно советуют всем, кто использует
СПО для построения своих программных продуктов, учитывать особенности лицензирования не только самих 
пакетов, но и входящих в их состав библиотек. Если вы используете copyleft библиотеку, 
это обязывает вас распространять свою программу под аналогичной лицензией,  т.е. любой человек, который
получит вашу программу легальным способом, может потребовать предъявить ему исходный код.

В работе над пособием принимал участие коллектив <<Базальт СПО>>:\\
М.\,А.\,Фоканова, М.\,О.\,Алексеева, А.\,А.\,Калинин, И.\,А.\,Мельников, Д.\,Н.\,Воропаев, А.\,А.\,Лимачко, В.\,А.\,Синельников, В.\,А.\,Соколов, В.\,П.\,Кашицин, А.\,В.\,Абрамов. Под редакцией В.\,Л.\,Чёрного. 

