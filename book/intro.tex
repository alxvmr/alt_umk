\chapter*{Введение}\addcontentsline{toc}{chapter}{Введение}
\markboth{Введение}{Введение}

Учебно-методическое пособие «Инфраструктура разработки программных 
пакетов и сборки программного обеспечения» предназначено для проведения 
занятий в очном и дистанционном режиме обучения в рамках гуманитарно-технологической 
платформы для целей обучения по программам магистратуры и дополнительного 
профессионального образования «Информационная культура цифровой трансформации».

<TO DO>
Пособие предназначено преподавателям и студентам .... 
Его можно применять для проведения семинаров и практикумов ...

Пособие состоит из ....

<END TO DO>

Пособие основано на реальном проекте разработки Сизиф и его стабильных бранчах (ветках), 
используемых ООО <<Базальт СПО>> для выпуска и обеспечения жизненного цикла коммерческих 
дистрибутивов. Часть из них включена в \href{reestr.digital.gov.ru}{Единый реестр российских программ} для электронных 
вычислительных машин и баз данных --- это Альт СП имеющий сертификаты ФСТЭК России, 
Минобороны России и ФСБ России, Альт Виртуализация, Альт Сервер, Альт Рабочая станция, 
Альт Образование, так и другие, бесплатные и свободные: Simply Linux, различные стартеркиты 
(Starterkits) и регулярные (regular) сборки. Дистрибутив --- это составное произведение, 
в составе которого есть программа для дистрибуции (установки), называемая инсталлятор и 
набор системного и прикладного ПО. В основе всех дистрибутивов лежат пакеты свободного 
программного обеспечения.

Свободное программное обеспечение (СПО) --- это программное обеспечение, распространяемое 
на условиях простой (неисключительной) лицензии, которая позволяет пользователю:
\begin{enumerate}
    \item использовать программу для ЭВМ в любых, не запрещённых законом целях;
    \item получать доступ к исходным текстам (кодам) программы как в целях изучения и адаптации, 
    так и в целях переработки программы для ЭВМ; распространять программу (бесплатно или за плату, по своему усмотрению);
    \item вносить изменения в программу для ЭВМ (перерабатывать) и распространять экземпляры изменённой (переработанной) 
    программы с учётом возможных требований наследования лицензии;
    \item в отдельных случаях (CopyLeft лицензия) распространять модифицированную компьютерную программу пользователем 
    на условиях, идентичных тем, на которых ему предоставлена исходная программа.
\end{enumerate}

Примерами свободных лицензий являются:
\begin{enumerate}
    \item \Sys{GNU general public license}. Version 3, 29 June 2007 (Стандартная общественная лицензия GNU. Версия 3, от 29 июня 2007 г.)
    \item \Sys{BSD license}, New Berkley Software Distribution license (Модифицированная программная лицензия университета Беркли)
\end{enumerate}

СПО отлично подходит для целей обучения и для разработки собственных решений, потому, что весь 
код доступен для изучения и модификации, однако авторы настоятельно советуют всем, кто использует 
СПО для построения своих программных продуктов учитывать особенности лицензирования не только самих 
пакетов, но и входящих в их состав библиотек, так как если вы используете copyleft библиотеку, 
это обязывает вас распространять свою программу под аналогичной лицензией --- любой человек, который 
поучит вашу программу легальным способом может потребовать предъявить ему исходный код.

