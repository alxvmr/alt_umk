\hypertarget{rpmbuild}{\chapter{Инструмент rpmbuild}}\label{chapter-rpmbuild}

\markboth{Инструмент rpmbuild}{Инструмент rpmbuild }

\Emph{rpmbuild} --- утилита сборки \Sys{rpm}-пакета из набора подготовленных файлов. Сборка осуществляется
в локальном окружении операционной системы, то есть в процессе сборки используются пакеты,
установленные в системе. rpmbuild позволяет собрать пакет с исходными данными (или в частном
случае пакеты с исходным кодом программ) и пакеты с бинарным содержимым.

Для работы с утилитой необходимо установить пакет rpm-build:

\begin{verbatim}
$ apt-get install rpm-build
\end{verbatim}

Общий вид структуры вызова утилиты:

\begin{verbatim}
$ rpmbuild [параметры]
\end{verbatim}

Для подробной информации вызовите справку командой:

\begin{verbatim}
$ rpmbuild ----help
\end{verbatim}

\section{Описание инструмента и сборка rpmbuild}

Параметры сборки \Emph{rpmbuild}:
\begin{itemize}
    \item \Sys{-ba} --- сборка исходного и двоичного пакета по файлу спецификации;
    \item \Sys{-bb} --- сборка только бинарного пакета по файлу спецификации;
    \item \Sys{-bs} --- сборка только пакета c исходными данными (src.rpm) по файлу спецификации;
    \item \Sys{-tb} --- сборка бинарного пакета из \Sys{tar}-архива;
    \item \Sys{-ts} --- сборка исходного пакета из \Sys{tar}-архива;
    \item \Sys{-bl} --- проверить список файлов в каталогах дерева окружения и вывести ошибки,
    если отсутствуют необходимые файлы;
    \item \Sys{----rmsource} --- удаление исходных файлов после завершения сборки;
    \item \Sys{----rmspec} --- удаление файла спецификации после завершения сборки.
\end{itemize}

Утилита предполагает наличие подготовленной структуры каталогов для сборки --- дерево каталогов.
(см.\,\hyperlink{3.4}{3.4} <<Рабочее пространство для сборки RPM-пакетов>>). По умолчанию используется
каталог \Sys{$\sim$/RPM}.

\begin{verbatim}
$ tree ~/RPM
/home/user/RPM
├── BUILD
├── RPMS
├── SOURCES
├── SPECS
└── SRPMS

5 directories, 0 files
\end{verbatim}

Выбранные для сборки исходные данные необходимо упаковать в \Sys{.tar}-архив:

\begin{verbatim}
$ tar -cvf <ИМЯ ФАЙЛА АРХИВА>.tar <КАТАЛОГ ИСХОДНЫХ ДАННЫХ>/
\end{verbatim}

Архив с исходными данными для сборки нужно поместить в каталог:\\ \Sys{./RPM/SOURCES/}.

Подготовленный \Sys{.spec} файл с инструкциями поместить в каталог: \\\Sys{./RPM/SPECS}.

\begin{verbatim}
$ cp <ИМЯ ФАЙЛА АРХИВА>.tar ~/RPM/SOURCES/<ИМЯ ФАЙЛА АРХИВА>.tar && \
cp <ИМЯ SPEC-ФАЙЛА>.spec ./RPM/SPECS/<ИМЯ SPEC-ФАЙЛА>.spec
\end{verbatim}

Пример сборки бинарного пакета \Sys{.rpm}:
\begin{verbatim}
$ rpmbuild -bb ~/RPM/SPECS/<ИМЯ SPEC-ФАЙЛА>.spec
\end{verbatim}
После успешного выполнения процесса сборки бинарные пакеты \Sys{.rpm} помещаются в каталог \Sys{$\sim$/RPM/RPMS/}.

\hypertarget{rpmbuild-exampl-src}{Пример сборки пакета с исходными данными \Sys{.src.rpm}}:
\begin{verbatim}
$ rpmbuild -bs ~/RPM/SPECS/<ИМЯ SPEC-ФАЙЛА>.spec
\end{verbatim}
После успешного выполнения процесса сборки, пакеты \Sys{.src.rpm} помещаются в каталог \Sys{$\sim$/RPM/SRPMS/}.

Рассмотрены два базовых сценария сборки пакета. Для получения различных результатов можно комбинировать ключи.
Например, для сборки и бинарного пакета и пакета с исходными данными можно использовать ключ \Sys{-ba}.
Полученные пакеты можно переносить и инсталлировать на различные компьютеры любым доступным способом.
В общем виде сборка пакета состоит из этапов:
\begin{itemize}
\item установка пакета \Sys{rpm-build};
\item подготовка окружения сборки;
\item формирование исходных данных;
\item упаковка исходных данных для сборки в \Sys{.tar}-архив;
\item подготовка файла спецификации для сборки \Sys{.spec};
\item размещение подготовленных файлов в каталогах окружения сборки;
\item запуск локальной сборки.
\end{itemize}

Сборка пакетов в локальном окружении операционной системы вызывает некоторые неудобства.
Приходится вручную раскладывать файлы в структуре каталогов \Sys{$\sim$/RPM}. При изменении
исходных данных приходится заново создавать архив. Работа в локальном окружении потенциально
может вызвать проблемы безопасности в случае сборки и исполнения заранее не проверенных исходных
данных собираемых программ и компонентов. Все дополнительные компоненты, необходимые
в процессе сборки, следует устанавливать в операционную систему так же в ручную.
В случае разработки дистрибутива операционной системы такие недостатки неприемлемы.
Поэтому разработчики дистрибутивов операционных систем создают свои собственные инструменты,
позволяющие исключить недостатки локальной сборки пакетов.

\section{Вопросы для самопроверки}

\begin{enumerate}
	\item Что такое \Sys{rpmbuild}, для чего предназначен?
	\item Какая структура каталогов необходима для начала сборки пакета средствами \Sys{rpmbuild}?
	\item Какие основные файлы необходимы для начала сборки пакета средствами \Sys{rpmbuild}?
	\item Какие два вида пакетов можно получить средствами \Sys{rpmbuild}?
	\item Перечислите основные этапы сборки пакета.
\end{enumerate}
